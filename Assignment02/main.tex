% !TEX program = xelatex

\documentclass[12pt, answers]{exam}

\usepackage{mathtools}
\usepackage[MnSymbol]{mathspec}
\usepackage{titling}
\usepackage{geometry}
\usepackage{tikz}
\usetikzlibrary{arrows, decorations.markings}
\usepackage{pgfplots}
\pgfplotsset{compat=newest}

\newgeometry{hmargin={12mm,17mm}}

\setmainfont[Numbers={Lining,Proportional}]{Minion Pro}
\setmathsfont(Digits,Latin,Greek)[Numbers={Lining,Proportional}]{Minion Pro}

\DeclareMathAlphabet{\mathbb}{U}{msb}{m}{n}

\pagestyle{headandfoot}
\runningheadrule
\footrule
\extraheadheight{-.3in}
\extrafootheight{-.75in}
\firstpageheader{}{}{}
\runningheader{\large\bfseries MT451P}{\large\bfseries Dheeraj Putta}{\large\bfseries Assignment 2}
\footer{}{Page \thepage\ of \numpages}{}

\setlength{\droptitle}{-5em}

\title{MT451P - Assignment 2}
\author{Dheeraj Putta \\ 15329966}
\date{}
\newcommand{\norm}[1]{\left\lVert#1\right\rVert}
\renewcommand{\d}{\mathrm{d}}

\linespread{1.15}

\begin{document}
    \maketitle
    \begin{questions}
        \thispagestyle{foot}
        \question \begin{parts}
            \part Find a unit speed parametrisation of the generalised helix, where $a, b > 0$ : \[ \alpha (t) = (a\cos{t}, a\sin{t}, bt) \]
            \begin{solution}
                We have that \[\alpha'(u) = (-a\sin{u}, a\cos{u}, b)\] and
                \begin{align*}
                    |\alpha'(u)| &= \sqrt{ (-a\sin{u})^2 + (a\cos{u})^2 + b^2)} \\
                    &= \sqrt{a^2\sin^2 u + a^2 \cos^2 u + b^2} \\
                    &= \sqrt{a^2(\sin^2 u + \cos^2 u) + b^2} \\
                    &= \sqrt{a^2 + b^2}
                \end{align*}
                Suppose that the original function was defined on the interval $[x, y]$. We now get that
                \begin{align*}
                    s(t) = \int_x^t \left|\alpha(u)\right| \;\mathrm{d}u = \int_x^t \sqrt{a^2 + b^2} \;\mathrm{d}u = t\sqrt{a^2 + b^2} - x\sqrt{a^2 + b^2}
                \end{align*}
                and the inverse of this function is
                \begin{align*}
                    t(s) = x+\frac{s}{\sqrt{a^2+b^2}}
                \end{align*}
                So the unit speed parameterisation is
                \begin{align*}
                    \Tilde{\alpha}(s) = \alpha(t(s)) = \left(a\cos\left(x + \frac{s}{\sqrt{a^2+b^2}}\right), a\sin\left(x + \frac{s}{\sqrt{a^2+b^2}}\right), \frac{bs}{\sqrt{a^2+b^2}} + bx\right)
                \end{align*}
                and the function is defined as $\Tilde{\alpha}: [0, s(y)] \to \mathbb{R}^3$
            \end{solution}

            \part Compute the acceleration vector to this unit speed parameterisation.

            \begin{solution}
            We have that \[\Tilde{\alpha}(s) = \left(a\cos\left(x + \frac{s}{\sqrt{a^2+b^2}}\right), a\sin\left(x + \frac{s}{\sqrt{a^2+b^2}}\right), \frac{bs}{\sqrt{a^2+b^2}} + bx\right)\]

            The tangent vector field is equal to
            \begin{align*}
                \Tilde{\alpha}'(s) = \left(\frac{-a\sin\left( x + \frac{s}{\sqrt{a^2 + b^2}}\right)}{\sqrt{a^2+b^2}}, \frac{a\cos\left( x + \frac{s}{\sqrt{a^2 + b^2}}\right)}{\sqrt{a^2+b^2}},\frac{b}{\sqrt{a^2+b^2}}\right)
            \end{align*}

            The acceleration vector field is equal to
            \begin{align*}
                \Tilde{\alpha}''(s) = \left( \frac{-a\cos\left( x + \frac{s}{\sqrt{a^2 + b^2}}\right)}{a^2+b^2}, \frac{-a\sin\left( x + \frac{s}{\sqrt{a^2 + b^2}}\right)}{a^2+b^2}, 0 \right)
            \end{align*}
            \end{solution}

            \part Sketch a picture showing the curve and its unit tangent and acceleration vector fields.
                \begin{solution}
                    \begin{center}
                        \begin{tikzpicture}[x=0.5cm,y=0.5cm,z=0.3cm,>=stealth]
                            % % The axes
                            % \draw[<->] (xyz cs:x=-11) -- (xyz cs:x=11) node[above] {$y$};
                            % \draw[<->] (xyz cs:y=-11) -- (xyz cs:y=11) node[right] {$z$};
                            % \draw[<->] (xyz cs:z=-11) -- (xyz cs:z=11) node[above] {$x$};
                            \begin{axis}[
                                view={60}{30},
                                axis line style = ultra thick,
                                axis lines=center,
                                height=11cm,
                                width=12cm,
                                xtick=\empty,
                                ytick=\empty,
                                ztick=\empty,
                                x label style={anchor=north},
                                  xlabel={$x$},
                                y label style={anchor=north},
                                  ylabel={$y$},
                                % z label style={anchor=north},
                                  zlabel={$z$},
                               ]
                               \addplot3+[domain=0:5*pi,samples=200,samples y=0, no marks, black, ultra thick]
                                ({sin(deg(x))},
                                {cos(deg(x))},
                                {2*x/(5*pi)});
                            \end{axis}
                        \end{tikzpicture}
                    \end{center}
                \end{solution}
        \end{parts}

        \newpage

        \question \begin{parts}
        \part Show that the curve given by
        \[ \beta(s) = \left( \frac{(1+s)^{3/2}}{3}, \frac{(1-s)^{3/2}}{3}, \frac{s}{\sqrt{2}} \right) \]
        defined on $s \in (-1, 1)$ is unit speed.

        \begin{solution}
            We have that
            \[ \beta'(s) = \left( \frac{\sqrt{1+s}}{2}, -\frac{\sqrt{1-s}}{2}, \frac{1}{\sqrt{2}}\right) \]
            Then
            \begin{align*}
                \left| \beta'(s) \right| &= \sqrt{\left(\frac{\sqrt{1+s}}{2}\right)^2 + \left(-\frac{\sqrt{1-s}}{2}\right)^2 + \left(\frac{1}{\sqrt{2}}\right)^2} \\
                &= \frac{1 + s + 1 - s + 2}{4}\\
                &= 1
            \end{align*}
            Therefore $\gamma$ is unit speed.
        \end{solution}

        \part Compute its Frenet frame.
        \begin{solution}
            From above we have that
            \begin{align*}
                T(s) &= \beta'(s) \\
                    &= \left( \frac{\sqrt{1+s}}{2}, -\frac{\sqrt{1-s}}{2}, \frac{1}{\sqrt{2}}\right)
            \end{align*}


            Then we have that
            \[
                T'(s) = \left( \frac{1}{4\sqrt{1 + s}}, \frac{1}{4\sqrt{1 - s}}, 0 \right)
            \]
            and
            \begin{align*}
                |T'(s)| &= \sqrt{ \left(\frac{1}{4\sqrt{1 + s}}\right)^2 + \left(\frac{1}{4\sqrt{1 - s}}\right)^2}\\
                &= \sqrt{\frac{1}{16+16s} + \frac{1}{16-16s}} \\
                &= \sqrt{\frac{16 - 16s + 16s + 16}{(16-16s)(16+16s)}} \\
                &= \frac{1}{\sqrt{8 - 8s^2}} \\
                &= \frac{1}{2\sqrt{2(1-s^2)}}
            \end{align*}
            So we get that
            \begin{align*}
                N(s) &= \frac{T'(s)}{|T'(s)|} \\
                &= \left( \frac{1}{4\sqrt{1 + s}}, \frac{1}{4\sqrt{1 - s}}, 0 \right) \cdot 2\sqrt{2(1-s^2)} \\
                &= \left( \frac{\sqrt{2(1-s^2)}}{2\sqrt{1 + s}}, \frac{\sqrt{2(1-s^2)}}{2\sqrt{1 - s}}, 0 \right) \\
                &= \left( \frac{\sqrt{-(1-s)}}{\sqrt{2}}, \frac{\sqrt{-(1+s)}}{\sqrt{2}}, 0\right)
            \end{align*}

            We know that
            \begin{align*}
                B(s) &= T(s) \times N(s) \\
                &= \left( \frac{\sqrt{1+s}}{2}, -\frac{\sqrt{1-s}}{2}, \frac{1}{\sqrt{2}}\right) \times \left( \frac{\sqrt{-(1-s)}}{\sqrt{2}}, \frac{\sqrt{-(1+s)}}{\sqrt{2}}, 0\right) \\
                &= \left(0, 0, \frac{\sqrt{(-1-s)(-s+1)} - \sqrt{(-1+s)(s+1)}}{4\sqrt{2(1-s^2)}}\right)
            \end{align*}
        \end{solution}
        \end{parts}

        \pagebreak

        \question \begin{parts}
            \part Show that the curve given by
        \[ \gamma(s) = \left( \frac{4}{5}\cos{s}, 1 - \sin{s}, -\frac{3}{5}\cos{s} \right) \]
        determines a circle.
            \begin{solution}
                We have that
            \begin{align*}
                T(s) &= \gamma'(s) \\
                    &= \left( -\frac{4}{5}\sin{s}, -\cos{s}, \frac{3}{5}\sin{s} \right)
            \end{align*}

            Then we have that
            \begin{align*}
                T'(s) = \left( -\frac{4}{5}\cos{s}, \sin{s}, \frac{3}{5}\cos{s} \right)
            \end{align*}
            and
            \begin{align*}
                |T'(s)| &= \sqrt{ \left(-\frac{4}{5}\cos{s}\right)^2 + \left( \sin{s}\right)^2 + \left(\frac{3}{5}\cos{s}\right)^2} \\
                &= \sqrt{\cos^2 s + \sin^2 s} \\
                &= 1 \\
                &= \kappa(s)
            \end{align*}
            So we get that $N(s) = T'(s)$ \\

            We know that
            \begin{align*}
                B(s) &= T(s) \times N(s) \\
                &= \left( -\frac{4}{5}\sin{s}, -\cos{s}, \frac{3}{5}\sin{s} \right) \times \left( -\frac{4}{5}\cos{s}, \sin{s}, \frac{3}{5}\cos{s} \right) \\
                &= \left( \frac{-3}{5}, 0, \frac{-4}{5}\right)
            \end{align*}
            and
            \[
                B'(s) = (0, 0, 0)
            \]
            For the equation $B'(s) = -\tau(s)N(s)$ to hold it must be the case that $\tau = 0$. \\

            Since $\tau = 0$ and $\kappa = 1$ this curves must determine a circle.
            \end{solution}

            \pagebreak

            \part Find it's centre and radius.
            \begin{solution}
                We know that
                \[ r = \frac{1}{\kappa} = 1 \]
                and
                \begin{align*}
                    c &= \gamma + \frac{1}{\kappa}N \\
                    &= \left( \frac{4}{5}\cos{s}, 1 - \sin{s}, -\frac{3}{5}\cos{s} \right) + \left( -\frac{4}{5}\cos{s}, \sin{s}, \frac{3}{5}\cos{s} \right) \\
                    &= (0, 1, 0)
                \end{align*}
            \end{solution}
        \end{parts}

        \question Consider a unit speed curve $\alpha$, with $\kappa > 0$ and $\tau \ne 0$, which lies on the sphere centered at $c \in \mathbb{R}$ of radius $r$.
        \begin{parts}
            \part Show that
            \[ \alpha - c = -\rho N - \rho' \sigma B\]
            where $\rho = \frac{1}{\kappa}$ and $\sigma = \frac{1}{\tau}$.
            \begin{solution}
                We have that
                \[(\alpha - c)^2 = \alpha^2 -2c\alpha + c^2 = r^2\]
                Differentiating both sides w.r.t to $s$ gives us that
                \begin{align*}
                    2\alpha'\alpha -2c\alpha' &= 0 \\
                    2\alpha'(\alpha - c) &= 0\\
                    \alpha'(\alpha - c) &= 0 \\
                    T(\alpha - c) &= 0 &\text{as } T = \alpha'
                \end{align*}
                Differentiating the above again w.r.t $s$, we get that
                \begin{align*}
                    T'\alpha + T\alpha' - cT' &= 0\\
                    T'(\alpha - c) + T^2 &= 0
                \end{align*}
                Substituting $T' = \kappa N$, we get
                \begin{align*}
                    \kappa N(\alpha - c) + 1 &= 0 \\
                \end{align*}
                Differentiating again w.r.t $s$,
                \begin{align*}
                    \kappa'N(\alpha - c) + \kappa N'(\alpha - c) + \kappa N\alpha' &= 0 \\
                    \kappa'N(\alpha - c) + \kappa N'(\alpha - c) &= 0 &\text{as } N\alpha' = NT = 0\\
                    -\frac{\kappa'}{\kappa} + \kappa N'(\alpha - c) &= 0 &\text{as } N(\alpha - c) = -\frac{1}{\kappa}
                \end{align*}
                Substituting $N' = -\kappa T + \tau B$
                \begin{align*}
                    -\frac{\kappa'}{\kappa} - \kappa^2T(\alpha - c) + \kappa \tau B(\alpha - c) &= 0 \\
                    -\frac{\kappa'}{\kappa} + \kappa \tau B(\alpha - c) &= 0 &\text{as } T(\alpha - c) = 0 \\
                    B(\alpha - c) &= \frac{k'}{k^2\tau} \\
                    &= \left(\frac{1}{\kappa}\right)' \frac{1}{\tau}
                \end{align*}
                We now have the following equations:
                \begin{align*}
                    T(\alpha - c) &= 0 \\
                    N(\alpha - c) &= -\frac{1}{\kappa} \\
                    B(\alpha - c) &= \left(\frac{1}{\kappa}\right)' \frac{1}{\tau}
                \end{align*}
                These 3 equations express the components of the radial vector $\alpha - c$ in the orthonormal frame $T, N, B$, which gives us that
                \begin{align*}
                    \alpha - c &= -\frac{1}{\kappa} N + \left(\frac{1}{\kappa}\right)'\frac{1}{\tau} B \\
                    &= -\rho N - \rho'\sigma B
                \end{align*}
            \end{solution}

            \part Deduce a formula for $r$ in terms of $\rho$ and $\sigma$.
                \begin{solution}
                From the part above we know that
                    \begin{align*}
                        r^2 &= (\alpha - c)^2 \\
                        &= (-\rho N - \rho' \sigma B)^2
                    \end{align*}
                    Since $\norm{N} = \norm{B} = 1$ and $NB = 0$, the above equation reduces to
                    \begin{align*}
                        r^2 &= \frac{1}{\kappa^2} + \left( \left( \frac{1}{\kappa} \right)' \frac{1}{\tau} \right)^2 \\
                        &= \frac{1}{\kappa^2} + \left( -\frac{\kappa'}{\kappa^2}\frac{1}{\tau^2}\right) \\
                        &= \frac{\kappa'^2 + \kappa^2\tau^2}{\kappa^4\tau^2} \\
                        r &= \frac{1}{\kappa^2\tau}\sqrt{\kappa'^2 + \kappa^2\tau^2}
                    \end{align*}
                \end{solution}

            \part For an arbitrary unit speed curve with $\kappa > 0$ and $\tau \ne 0$, specify conditions on $\rho$ and $\sigma$ which would guarantee that $\alpha$ lies on a sphere of radius $r$.
            \begin{solution}
                We assert that the condition

        \[\frac{\tau}{\kappa} \equiv \frac{\d}{\d s} \left(\frac{\kappa'}{\kappa^2 \tau} \right) = -\frac{\d}{\d s} (\rho' \sigma) \]

    is sufficient to guarantee this property. Define
        \begin{align}
            c(s) &\coloneqq \alpha(s) + \rho(s)N(s) + \rho'(s)\sigma(s)B(s) \\
            r^2(s) &\coloneqq \rho(s)^2 + \rho'(s)^2 \sigma(s)^2
        \end{align}

    These will define the centre and the square of the radius of the circle, respectively, provided they are both constant. Differentiating $(1)$ with respect to $s$, we obtain
        \begin{align*}
            c' &= \alpha' + \rho' N + \rho N' + \frac{\d}{\d s}(\rho' \sigma)B + (\rho' \sigma)B' \\
            &= T + \rho'N + \rho(-\kappa T + \tau B) + \frac{\d}{\d s} (\rho' \sigma)B + (\rho' \sigma)(-\tau N) \\
            &= \left(\frac{\rho}{\sigma} + \frac{\d}{\d s}(\rho' \sigma) \right)B = \left(\frac{\tau}{\kappa} - \frac{\tau}{\kappa}\right)B \equiv 0
        \end{align*}

    And thus $c$ is constant, as required. Likewise, differentiating $(2)$ with respect to $s$, we also obtain
        \begin{align*}
            r^{2\prime} &= \frac{\d}{\d s} (\rho^2 + (\rho' \sigma)^2) = 2\rho\rho' + 2\rho' \sigma \frac{\d}{\d s}(\rho' \sigma) \\
            &= 2\rho\rho' - 2\rho'\sigma \frac{\rho}{\sigma} = 2\rho\rho' - 2\rho\rho' = 0
        \end{align*}

    So $r^2$, and thus $r$, is constant, as required. Overall, our initial assertion is sufficient to guarantee that $\alpha$ lie on a sphere with centre and radius as defined.
            \end{solution}
        \end{parts}
    \end{questions}
\end{document}
