% !TEX program = xelatex

\documentclass[12pt, answers]{exam}

\usepackage{mathtools}
\usepackage[MnSymbol]{mathspec}
\usepackage{titling}
\usepackage{geometry}


\newgeometry{hmargin={12mm,17mm}}

\setmainfont[Numbers={Lining,Proportional}]{Minion Pro}
\setmathsfont(Digits,Latin,Greek)[Numbers={Lining,Proportional}]{Minion Pro}

\DeclareMathAlphabet{\mathbb}{U}{msb}{m}{n}

\pagestyle{headandfoot}
\runningheadrule
\footrule
\extraheadheight{-.3in}
\extrafootheight{-.75in}
\firstpageheader{}{}{}
\runningheader{\large\bfseries MT451P}{\large\bfseries Dheeraj Putta}{\large\bfseries Assignment 3}
\footer{}{Page \thepage\ of \numpages}{}

\setlength{\droptitle}{-5em}

\title{MT451P - Assignment 3}
\author{Dheeraj Putta \\ 15329966}
\date{}
\newcommand{\norm}[1]{\left\lVert#1\right\rVert}
\renewcommand{\d}{\mathrm{d}}
\DeclareMathOperator{\im}{Im}

\linespread{1.15}

\begin{document}
    \maketitle
    \begin{questions}
        \thispagestyle{foot}

        \question Making use of the Implicit Function Theorem, derive the Regular Level Set Theorem.
        \begin{solution}
            Let $f : \mathbb{R}^2 \times \mathbb{R} \rightarrow \mathbb{R}$ be a smooth function. Let $c \in \im f$ and define
            $\Sigma \coloneqq f^{-1}(c) \subseteq \mathbb{R}^3$. Since $c$ is a regular value, we have that $\Sigma$ contains
            no critical points of $f$. For $p \in \Sigma$, we have that
            \[ Df_p = \left[\frac{\partial f}{\partial x_1}(p), \frac{\partial f}{\partial x_2}(p), \frac{\partial f}{\partial x_3}(p)\right] \]
            has a non-zero component. Then by the Implicit Function Theorem, $\exists U \subseteq \mathbb{R}^2, V \subseteq \mathbb{R}$
            with $p \in U \times V$, and a smooth function $g:U \rightarrow V$, such that $(u_1, u_2, g(u_1, u_2)) \in \Sigma$,
            $\forall u_1, u_2 \in U$. Then define
            \[ \tilde{x}: U \rightarrow R^3 :  (u_1, u_2) \to (u_1, u_2, g(u_1, u_2))\]
            Since $g$ is smooth this must mean that $\tilde{x}$ is smooth as well. Also, since $\tilde{x}(u_1, u_2)$ is
            uniquely determined by $(u_1, u_2)$, $\tilde{x}$ is injective. Both $\Sigma$ and $g$ vary continuously, which
            means that $g(p_1, p_2) = p_3$ and $(p_1, p_2, p_3) \in \im \tilde{x}$. By the Implicit Function Theorem we
            get that $\tilde{x}(u_1, u_2) \in \Sigma, \forall \, (u_1, u_2) \in U$, so $\im \tilde{x} \subseteq \Sigma$.
            We also know that,
            \[ \left[ \frac{\partial\tilde{x}_1}{\partial u_1}, \frac{\partial\tilde{x}_2}{\partial u_1}  \frac{\partial\tilde{x}_3}{\partial u_1}\right]
            \times \left[ \frac{\partial\tilde{x}_1}{\partial u_2}, \frac{\partial\tilde{x}_2}{\partial u_2}  \frac{\partial\tilde{x}_3}{\partial u_2}\right]
             = [1, 0, g_{u_1}] \times [0, 1, g_{u_2}] = [-g_{u_1}, g_{u_2}, 1] \neq 0\]
            that is, $\tilde{x}_{u_1} \times \tilde{x}_{u_2}$ is non-zero on all of $U$, which means that $\tilde{x}$ is
            regular. Since $\tilde{x}$ satisfies the conditions for being a co-ordinate patch containing $p$, and as this
            hold $\forall \, p \in \Sigma$, $\Sigma$ is a surface.
        \end{solution}
        \pagebreak
        \question Which of the following subsets of $\mathbb{R}^3$ are surfaces. Provide a brief justification for your answer in
        each case.

        \begin{parts}
            \part The solution set for the equation
            \[ \frac{1}{3}z^3 - z = \frac{1}{2}x^2 - \frac{1}{2}y^2\]

            \begin{solution}
                Define
                \[ f:\mathbb{R}^3 \rightarrow \mathbb{R}: (x, y, z) \to \frac{1}{2}x^2 - \frac{1}{2}y^2 - \frac{1}{3}z^3 + z\]
                Then
                \[ \nabla f = \left[x, y, 1 - z^2\right] \]
                The only points where $\nabla f = 0$ are $(0, 0, \pm 1)$ but since
                \[ f(0, 0, \pm 1) = \pm 1 \pm \frac{1}{3} \neq 0\]
                these points are not in $f^{-1}(0)$. Therefore, 0 is a regular value of $f$ and by the Regular Level Set
                Theorem, the set $f^{-1}(0)$ is a surface.
            \end{solution}

            \part The sphere $S^2 \subset \mathbb{R}^3$ consisting of all points in $\mathbb{R}^3$ whose distance from the
            origin is 1.

            \begin{solution}
                Define
                \[ f: \mathbb{R}^3 \to \mathbb{R} : (x,y,z) \to (x^2 + y^2 + z^2)\]
                Then
                \[ \nabla f = \left[ 2x, 2y, 2z \right] \]
                The only point where $\nabla f = 0$ is at $(0, 0, 0)$ but since
                \[ f(0, 0, 0) = 0 \neq 1\]
                it is not in $f^{-1}(1)$. Therefore 1 is a regular value of $f$ and by the Regular Level Set
                Theorem, this set is a surface.
            \end{solution}

            \part The set of points $(x, y, f(x, y))$ where $f: \mathbb{R}^2 \rightarrow \mathbb{R}$ is a smooth function.

            \begin{solution}
                Define
                \[ \tilde{x}: \mathbb{R}^2 \rightarrow \mathbb{R}^3 : (x, y) \to (x, y, f(x, y))\]
                This is a smooth function as $f$ is a smooth function. It is also an injective function as $\tilde{x}$ is
                uniquely determined by $(x, y)$. Then we get that
                \[  \left[ \frac{\partial\tilde{x}_1}{\partial x}, \frac{\partial\tilde{x}_2}{\partial x}  \frac{\partial\tilde{x}_3}{\partial x}\right]
                \times \left[ \frac{\partial\tilde{x}_1}{\partial y}, \frac{\partial\tilde{x}_2}{\partial y}  \frac{\partial\tilde{x}_3}{\partial y}\right]
                 = [1, 0, f_x] \times [0, 1, f_y] = [-f_x, f_y, 1] \neq \vec{0} \]
                 Therefore $\tilde{x}_x \times \tilde{x}_y$ is non-zero everywhere and so it is regular. This means that
                 $\tilde{x}$ is a co-ordinate patch whose domain is all of $\mathbb{R}^2$, meaning it suffices for all inputs
                 and its image is the set we seek. Theorem that set is a surface.
            \end{solution}

            \part The subset of $\mathbb{R}^3$ obtained upon revolution about the $z$-axis of the circle given by the
            equation
            \[ (x - R)^2 + z^2 = r^2 \]
            where $R, r > 0$ are constants.

            \begin{solution}
                We can define this subset by using cylindrical co-ordinates. Define
                \[ f:\mathbb{R}^3 \rightarrow \mathbb{R} : (d, \theta, z) \to (d-R)^2 + z^2 = d^2 - 2Rd + R^2 + z^2\]
                Then we get that
                \[ \nabla f = \left[0, 2d - 2R, 2z \right] \]
                This is only zero points of the form $(R, \theta, 0)$ for $\theta \in [0, 2\pi)$. But
                $f(R, \theta, 0) = (R - R)^2 + 0^2 = 0 \neq r^2$ as $r > 0$. This means that $r^2$ is a regular value of $f$.
                Since $f^{-1}(r^2)$ contains none of the critical points, by the Level Set Theorem, it is a surface.
            \end{solution}

            \part The set of all $(x, y, z) \in \mathbb{R}^3$ which satisfy the equation $x^2 + z^2 = y^2$.

            \begin{solution}
                The set described y this equation is the double cone connected at the origin. Suppose this set is a surface.
                This means there is a co-ordinate patch $\tilde{x}$ which maps from an open area of $\mathbb{R}^2$ to an
                open neighbourhood of the origin in $\mathbb{R}^3$.\\\\
                Since $\tilde{x}$ is a homeomorphism this would mean that it would preserve the connectedness property of
                open sets in $\mathbb{R}^2$. However removing the origin in $\mathbb{R}^3$ would leave us with two disjoint
                cones. Therefore such an $\tilde{x}$ cannot exist and as such the set described above is not be a surface.
            \end{solution}

            \part The union of the sets $A$ and $B$ where
            \begin{align*}
                A = \{(x, y, 0) : x^2 + y^2 < 1\} \quad \text{and} \quad B = \{ (x, y, z) : x^2 + y^2 + z^2 = 1 \}
            \end{align*}

            \begin{solution}
                The union of these two sets is the sphere of radius 1 with a disk inside on the $xy$-plane. The area where
                these two sets intersect is the unit circle on the $xy$-plane. An open neighbourhood on the unit circle
                will not be locally euclidean, so it must be that $A \cup B$ is not a surface.

            \end{solution}

            \part The union of the sets $A$ and $B$ where
            \begin{align*}
                A = \{(x, y, 0) : x^2 + y^2 < 1\} \quad \text{and} \quad B = \{ (x, 0, z) : x^2 + z^2 < 1 \}
            \end{align*}

            \begin{solution}
                This set describes the disks on the $xy$ and $xz$ plane which intersect on the $x$ plane. If we consider an
                open neighbourhood around $(1, 0, 0)$ with that point removed then it is not connected. Then by the same reason
                as in (e), this set cannot be a surface.
            \end{solution}

            \part The intersection of the sets $A$ and $B$ where
            \begin{align*}
                A = \{(x, y, z) : 3x^2 + 7y^2 < 1\} \quad \text{and} \quad B = \{ (x, y, 0) : x, y > 0 \}
            \end{align*}

            \begin{solution}
                The intersection of these two sets is the set $C = \{ (x, y, 0) : 3x^2 + 7y^2 < 1 \text{ and } x, y > 0\}$,
                which is the positive quadrant of an ellipse without boundary. Since this set has no critical points and is
                an open region of $\mathbb{R}^2$, it is a surface.
            \end{solution}
        \end{parts}

    \end{questions}
\end{document}
