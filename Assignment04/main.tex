% !TEX program = xelatex

\documentclass[12pt, answers]{exam}

\usepackage{mathtools}
\usepackage[MnSymbol]{mathspec}
\usepackage{titling}
\usepackage{geometry}


\newgeometry{hmargin={12mm,17mm}}

\setmainfont[Numbers={Lining,Proportional}]{Minion Pro}
\setmathsfont(Digits,Latin,Greek)[Numbers={Lining,Proportional}]{Minion Pro}

\DeclareMathAlphabet{\mathbb}{U}{msb}{m}{n}

\pagestyle{headandfoot}
\runningheadrule
\footrule
\extraheadheight{-.3in}
\extrafootheight{-.75in}
\firstpageheader{}{}{}
\runningheader{\large\bfseries MT451P}{\large\bfseries Dheeraj Putta}{\large\bfseries Assignment 4}
\footer{}{Page \thepage\ of \numpages}{}

\setlength{\droptitle}{-5em}

\title{MT451P - Assignment 4}
\author{Dheeraj Putta \\ 15329966}
\date{}
\newcommand{\norm}[1]{\left\lVert#1\right\rVert}
\renewcommand{\d}{\mathrm{d}}
\DeclareMathOperator{\im}{Im}

\linespread{1.15}

\begin{document}
    \maketitle
    \begin{questions}
        \thispagestyle{foot}
        \question Compute the shape operator and Gaussian and mean curvatures for the round sphere of radius $r$, the round
        cylinder of radius $r$ and any flat plane in $R^3$.

        \begin{solution}
            asd
        \end{solution}

        \question Assume that the surface $\Sigma$ is described locally near $p$ as the graph of the function
        $f: \mathbb{R}^2 \rightarrow \mathbb{R}$, with $p = (0, 0, 0)$. Further suppose that $f$ satisfies
        \[ f_x(0, 0) = f_y(0, 0) = 0 \]
        Then prove that the shape operator, $S$, of $\Sigma$ at $p = (0, 0, 0)$, has matrix (in the standard basis)
        of the following form:
        \[
            \text{Matrix } S =
            \begin{bmatrix}
                f_{xx}(0, 0) & f_{xy}(0, 0) \\
                f_{yx}(0, 0) & f_{yy}(0, 0)
            \end{bmatrix}
        \]

        \begin{solution}
            A local parametrization around around $p$ is given by
            \[ \varphi:(u, v) \to (u, v, f(u, v)) \]
            Then the normal vector field is given by
            \[ N = \frac{\varphi_u \times \varphi_v}{|\varphi_u \times \varphi_v|} = \frac{(-f_u, -f_vm, 1)}{\sqrt{1 + f_u^2 + f_v^2}} \]
            Define $\alpha = \sqrt{1 + f_u^2 + f_v^2}$. Then
            \begin{align*}
                \nabla_{\vec{e_1}}N &= \left( \frac{\partial}{\partial u} \left(\frac{-f_u}{\alpha}\right),
                \frac{\partial}{\partial u} \left(\frac{-f_v}{\alpha}\right), \frac{\partial}{\partial u} \left(\frac{1}{\alpha}\right) \right)\\
                \nabla_{\vec{e_2}}N &= \left( \frac{\partial}{\partial v} \left(\frac{-f_u}{\alpha}\right),
                \frac{\partial}{\partial v} \left(\frac{-f_u}{\alpha}\right), \frac{\partial}{\partial v} \left(\frac{1}{\alpha}\right) \right)
            \end{align*}
            The derivative will be similar for all of them so we will only work out 1 of them.
            \[ \frac{\partial}{\partial u}\bigg\rvert_{p = 0} \left( \frac{-f_u}{\alpha} \right) =
            \frac{-\alpha f_{uu} + \alpha'f_u}{\alpha^2}\bigg\rvert_{p = } = -f_{uu}(0, 0)\]
            Finally we get that
            \[-\nabla_{v_1\vec{e}_1 + v_2\vec{e}_2}N = v_1\left( f_{uu}(0, 0), f_{uv}(0, 0), 0 \right) +
            v_2\left( f_{vu}(0, 0), f_{vv}(0, 0), 0 \right)\]
            which can be represented as
            \[
                \begin{bmatrix}
                    f_{xx}(0, 0) & f_{xy}(0, 0) \\
                    f_{yx}(0, 0) & f_{yy}(0, 0)
                \end{bmatrix}
            \]
        \end{solution}
    \end{questions}
\end{document}
